# Hardwarenahe_Programmierung
\documentclass[11pt,a4paper,draft]{article}
\usepackage[utf8]{inputenc}
\usepackage[german]{babel}
\usepackage[T1]{fontenc}
\usepackage{amsmath}
\usepackage{amsfonts}
\usepackage{amssymb}
\usepackage{graphicx}
\usepackage{minted}
\author{Ansgar Dabow}
\begin{document}
\tableofcontents
\newpage

\section{Praktikum}
mikrocontroller.net für weitere Infos (Schaltpläne lesen)

\subsection{Umgang mit der Software(KEIL)}

erst übersetzen(F7),
dann runterladen(F8)

\subsection{Aufgabe 2}
\begin{list}{•}{}
\item in der PDF von chip1768 den Schaltplan aufschlagen und in die GPIOs sehen. In GPIO(1) finden sich die LEDs. Dort sind die Ports mit $P[x]$ dargestellt, wobei x das Bit für den jeweiligen Port darstellt.
\item Über \#DEFINE die LEDx mit der zugehörigen Bitkonstanten verkoppeln 
\end{list}

\begin{list}{•}{}
\item mit FIODIR die Ports auf Eingang setzen \\ $(LPC_GPIO1->FIODIR |= 1<<LEDx)$
\item mit FIOSET Strom auf den Port geben 		(=1)
\item mit FIOCLR Strom auf den Port abstellen 	(=1)
\item FIOPIN gibt den Status der LED an, somit kann auch direkt auf die LED zugegriffen werden
\end{list}

\subsection{Aufgabe 4}
\begin{list}{•}{}
\item Achtung, LEDs 4 und 5 gehen nur auf LOW an!
\end{list}

\subsection{code}
\begin{minted}{c}
#include "LPC17xx.h"

#define LED2 20
#define LED3 21
#define SPEAKER 5
#define LED4 8
#define LED5 9

void delay(int ms)
{
	int y = 0;
	int z = 0;
	while (z < ms)
	{
		z++;
		y=0;
		while (y<20000)
		{
			y++;
		}
	}
}

void testBedLedBlink(int led)
{
	delay(200);
	LPC_GPIO0->FIOCLR |= 1<<led;
	delay(200);
	LPC_GPIO0->FIOSET |= 1<<led;
}

int main()
{

	//Initialisierungen
	LPC_GPIO1->FIODIR |= 1<<LED2;
	LPC_GPIO1->FIODIR |= 1<<LED3;
	LPC_GPIO0->FIODIR |= 1<<LED4;
	LPC_GPIO0->FIODIR |= 1<<LED5;
	LPC_GPIO2->FIODIR |= 1<<SPEAKER;

	while(1)
	{
		/* 	Aufgabe 1
		LPC_GPIO0->FIOSET |= 1<<LED4;
		delay(100);
		LPC_GPIO0->FIOCLR |= 1<<LED4;
		delay(100);
		/**/
		/*	Aufgabe 2
		LPC_GPIO1->FIOCLR |= 1<<LED3;
		LPC_GPIO1->FIOSET |= 1<<LED2;
		delay(4000);
		LPC_GPIO1->FIOCLR |= 1<<LED2;
		LPC_GPIO1->FIOSET |= 1<<LED3;
		delay(4000);
		/**/
		/* 	Aufgabe 3
		LPC_GPIO2->FIOSET |= 1<<SPEAKER;
		delay(1);
		LPC_GPIO2->FIOCLR |= 1<<SPEAKER;
		delay(1);
		/**/
		/*	Aufgabe 4
		testBedLedBlink(LED4);
		testBedLedBlink(LED5);
		/**/
	}
}
\end{minted}
\section{Praktikum}
\subsection{TASTENDRÜCKEN}
\begin{list}{•}{Button abrufen}
\item $if(LPC\_GPIO0->FIOPIN \& (1<<BT1) == 0)$
\item BT1 = auf die Binärstelle des Button 1 gestellt
\item Wenn FIOPIN an der Stelle BT1 "gedrückt" ist, dann ist dort "Null" gesetzt
\item Wenn FIOPIN an der Stelle BT1 "nicht gedrückt" ist, dann ist dort "Eins" gesetzt
\item Bei der Und-Verknüpfung sollte an der Stelle von BT1 der Wert herauskommen der signalisiert ob der Button gedrückt ist
\end{list} 	

\subsection{CODE}
\begin{minted}{c}
#include "LPC17xx.h"                          
#include <stdbool.h>
//Definitionszuweisungen------------------
#define LED1 18   //P1.18 LPC Modul high-aktiv
#define LED2 20   //P1.20 LPC Modul high-aktiv
#define LED3 21   //P1.21 LPC Modul high-aktiv
#define LED4 23   //P1.23 LPC Modul high-aktiv
#define BT1 6     //Button 1 an P0.6
#define BT2 16    //Button 2 an P0.16
#define SPK 5     //Speaker P2.5
#define LED4_BD 9   //LED4 mBed low-aktiv
#define LED5_BD 8   //LED5 mBed low-aktiv
//Funktionsdeklarationen------------------
int warte();
int outLED();   //Ausgabe Zählvariable auf LED Zeile LPC Modul 

// Variablenzuweisungen-------------------
// globale variable
int izaehl=0;   //Zählvariable
//***Mainprogramm*******************************
int main()
{
 //Initialisierungen ---------------------
 LPC_GPIO1->FIODIR|=(1<<LED1);      //LED1 1-4 LPC Modul auf Portausgang
 LPC_GPIO1->FIODIR|=(1<<LED2);
 LPC_GPIO1->FIODIR|=(1<<LED3);
 LPC_GPIO1->FIODIR|=(1<<LED4);
 LPC_GPIO2->FIODIR|= 1<<SPK;
 
 bool toggle = 0;
 
 while(1)
 {
  /*
  if((LPC_GPIO0->FIOPIN & (1<<BT1)) == 0)
  {
   warte(50);
   if(toggle)
    toggle = 0;
   else
    toggle = 1;
   while((LPC_GPIO0->FIOPIN & (1<<BT1)) == 0);
  }
  if(toggle)
  {
   // Aufgabe 2
   LPC_GPIO1->FIOSET |= 1<<LED1;
   warte(500);
   LPC_GPIO1->FIOCLR |= 1<<LED1;
   warte(500);
   
   // Aufgabe 3
   LPC_GPIO2->FIOSET |= 1<<SPK;
   warte(1);
   LPC_GPIO2->FIOCLR |= 1<<SPK;
   warte(1);
  }
  */
  if((LPC_GPIO0->FIOPIN & (1<<BT1)) == 0)
   LPC_GPIO1->FIOSET |= 1<<LED1;
  if((LPC_GPIO0->FIOPIN & (1<<BT2)) == 0)
   LPC_GPIO1->FIOSET |= 1<<LED4;
  warte(500);
  LPC_GPIO1->FIOCLR |= 1<<LED1;
  LPC_GPIO1->FIOCLR |= 1<<LED4;
  warte(500);
 } 
}
 
//***Funktionen********************************* 
//Zeitschleife in Millisekunden
//-------------------------------
int warte(int izahl)   //izahl in Millisekunden
{
 int iz=0,iy=0;
 while (iz<izahl)
      {
      iz++;
      iy=0; 
      while (iy<20000)
            { 
         iy++;
            }    
      } // ende while
}
\end{minted}
\section{Praktikum}
\subsection{TIMER EINSTELLEN}
\begin{list}{•}{}
\item Timer zählt kontinuierlich hoch (max. 2³²)
\item Es wird ein "Match-Register" gesetzt, dass festsetzt bis wie weit der Timer zählt
\item Wenn der Timer mit dem Match-Register matched, wird ein Ereignis ausgelöst (z.B. ein Interrupt, Timer zurücksetzen Anhalten) -> Match-Control Register (MRxI/MRxR/MRxS x - Matchregister Nummer \#)
\item Der Timer fängt daraufhin wieder von vorne an zu zählen

\item die Hälfte der Frequenz gibt an, wie lange der Timer braucht um einmal bis zum Maximum hochzulaufen (Speed)

\item Formeln (siehe PR\_DOKU Seite 19)
\item PCLK gibt für jeden Takt der CPU einen Impuls
\begin{enumerate}
\item PCLK modifiziert werden (jede 2./4./8.)
\end{enumerate}
\item Jeder Timer hat eigene 4 Matchregister
\item Es gibt 4 Timer

\item PCLKSELx gibt den Teiler für die PCLK an\\
\begin{tabular}{c | c}
00	& /1.\\
01	& /2.\\
10	& /4. (Standard)\\
11	& /8.
\end{tabular}										 )
\end{list}

\subsection{ABLAUF}
\begin{enumerate}
\item Timer initialisieren
\begin{itemize}
\item Timer 0 einschalten
\begin{itemize}
\item $LPC\_SC->PCONP |=1<<1$
\end{itemize}
\item Clock-Teiler einstellen
\begin{itemize}
\item $LPC\_SC->PCLKSEL0 |=3<<2$
\end{itemize}
\item Matchregister einstellen
\begin{itemize}
\item $cpu takt = 100mhz$
\item $100mhz / 8 = 12,5$
\item $MR = 12,5Mhz/2*1hz$
\item $MR = 6250000$
\item $LPC\_TIM0->MR0 = 6250000$
\end{itemize}
\item Einstellen was das Matchregister tun soll
\begin{itemize}
\item item Interrupt soll ausgelöst werden
\item $LPC\_TIM0->MCR |= 3<<0$
\end{itemize}
\end{itemize}
\item Interrupt Service Routine Schreiben
\begin{itemize}
\item z.B. void $TIMER0\_IRQHandler()$
\item name ist vorgegeben und muss zwingend verwendet werden
\item $system\_LPC17xx.s$ definiert die Namen der ISR-Funktionen (External Interrupts)
\begin{itemize}
\item enthält die IVT (Interrupt Vektor Table)
\item Timer 0 meldet IMMER an Port 17
\item Ist über $system\_LPC17xx.s$ nachvollziehbar
\end{itemize}
\item !!!Interrupt Flags zurücksetzen!!!
\begin{itemize}
\item sonst bleibt der Interrupt bestehen und man würde nie die ISR verlassen
\item $LPC\_TIM0->IR |=1<<0$
\begin{itemize}
\item Setzen um zu resetten	
\item Interrupt Register um Bit-Positionen zu finden
\end{itemize}
\end{itemize}
\item Nutzfunktion implementieren
\end{itemize}
\item Interrupt Service Routine Priorität setzen
\begin{itemize}
\item $NVIC\_SetPriority(TIMER0\_IRQn,15)$
\end{itemize}	
\item Interrupt Service Routine aktivieren (sonst würde nie die Funktion abgefeuert werden)
\begin{itemize}
\item $NVIC\_EnableIRQ(TIMER0\_IRQn)$
\end{itemize}
\item Timer aktivieren
\begin{itemize}
\item$ LPC\_TIM0->TCR |=1<<0$
\end{itemize}
\end{enumerate}
\subsection{CODE}

\end{document}
